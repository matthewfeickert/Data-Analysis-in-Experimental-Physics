\documentclass[aps,letterpaper,12pt]{article}

\usepackage{graphicx} % For images
\graphicspath{{Images/}}
\usepackage{float}		% For tables and other floats
\usepackage{amsmath}  	% For math
\usepackage{amssymb}  	% For more math
\usepackage{fullpage} 	% Set margins and place page numbers at bottom center
\usepackage{subfig}   	% For subfigures
\usepackage[usenames,dvipsnames]{xcolor}	% For colors and names for color boxed links
%\usepackage[usenames,dvipsnames]{color}	% For colors and names
\usepackage[]{hyperref}           		% For hyperlinks and indexing the PDF
\hypersetup{
    colorlinks=false,		% Surround the links by color frames (false) or colors the text of the links (true)
    citecolor=blue,		% Color of citation links
    filecolor=black,		% Color of file links
    linkcolor=red,		% Color of internal links (sections, pages, etc.)
    urlcolor=black,		% Color of url hyperlinks
    linkbordercolor=red, 	% Color of links to bibliography
    citebordercolor=blue,	% Color of file links
    urlbordercolor=blue	% Color of external links
}
\usepackage{calligra}	% For generating 'script r'
\DeclareMathAlphabet{\mathcalligra}{T1}{calligra}{m}{n}
%\DeclareFontShape{T1}{calligra}{m}{n}{<->s*[2.2]callig15}{}
\usepackage{mathrsfs}	% For nice math calligraphy fonts
\usepackage{custom}
%\usepackage{draftmode}		% Draft mode edit;	Remember to remove the linenumbers command
\usepackage{soul}			% Allows strikethrough of text with \st{text}
\usepackage{cancel}		% Allows for \cancelto{cancel_to_value}{quantity_being_canceled}
\usepackage{mathtools}	% Needed for mathclap
\usepackage{tabu}

%\usepackage{al?phalph}
\renewcommand\vec[1]{\ensuremath\boldsymbol{#1}}	% Makes vectors boldface type, replaces arrows.
\renewcommand{\thefootnote}{\fnsymbol{footnote}}		% Causes footnotes to be indicated symbols, not numbers
%\numberwithin{equation}{section}					% Labels equations due to their section number
\hyphenation{ATLAS}							% Defines where words may be hyphenated

% Redefine the itemize symbol to be empty space for problem 3.
%\renewcommand*\labelitemi{}


% Change enumerate bullet point
\usepackage{enumerate}

\usepackage{caption}
\usepackage{empheq}

% 
\begin{document}
\begin{centering}
\textbf{Statistical Data Analysis in Experimental Physics}\\
\textbf{Syllabus}\\
\textbf{Semester Year}\\
\end{centering}\vspace{30pt}

\section*{Course Description}
A 12 week course focusing on the underlying theory of statistics that drives modern data analysis in experimental techniques in experimental physics. A practical understanding of the theory and techniques will be acquired by writing analysis code in \texttt{C++}, using the ROOT data analysis framework and RooFit package. The course will be built around the provided course notes.

\section*{Course Topics}

\begin{enumerate}
	\item A Review of Introductory Statistics
	\item \ldots
	\item Model Optimization
	\item \ldots
\end{enumerate}

\section*{Recommended Texts}

\begin{itemize}
	\item \href{http://smile.amazon.com/Data-Analysis-High-Energy-Physics/dp/3527410589/}{Data Analysis in High Energy Physics: A Practical Guide to Statistical Methods (1st Edition)}, \emph{Behnke, et al.}
	\item \href{http://smile.amazon.com/dp/3527410864/}{Statistical Analysis Techniques in Particle Physics: Fits, Density Estimation and Supervised Learning 1st Edition}, \emph{Narksy and Porter}
	\item The \href{https://root.cern.ch/root-user-guides-and-manuals}{RooFit Users Manual}
	\item \ldots
\end{itemize}

\section*{Grading}


%###########
%%%%%%%%
\end{document}
%%%%%%%%
%###########